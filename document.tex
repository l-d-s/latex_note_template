\documentclass[11pt]{article}
% [fleqn] left-aligns all equations

% fonts

\usepackage[T1]{fontenc} % Apparently this https://tex.stackexchange.com/questions/287312/font-not-loadable-metric-data-not-found-or-bad
\usepackage[full]{textcomp}
\usepackage{cabin} % sans serif
\usepackage[varqu,varl]{inconsolata} % sans serif typewriter
\usepackage[final,expansion=alltext]{microtype}
\usepackage[english]{babel}
\usepackage[fleqn]{mathtools}
\usepackage[cal=boondoxo]{mathalpha} % mathcal

\usepackage{newtxtext}
% \usepackage{txfonts} % See https://tex.stackexchange.com/questions/444243/symbols-not-showing-up


\usepackage[bigdelims,vvarbb]{newtxmath} % bb from STIX % "should be loaded AFTER the text font" - https://mirrors.mit.edu/CTAN/fonts/newtx/doc/newtxdoc.pdf 

% geometry of the page

\usepackage[
    top=1in,
    bottom=1in,
    left=1in,
    right=1in
    ]{geometry}

% paragraph spacing

\setlength{\parindent}{0pt}
\setlength{\parskip}{1.5ex plus 0.4ex minus 0.2ex}

% useful packages

% \usepackage{natbib} % incompatible with BibLaTeX
\usepackage{epsfig}
\usepackage{url}
\usepackage{bm}
\usepackage{blindtext}

% Leon additions

\usepackage{enumitem} % allows for enumerate[resume]
\usepackage[style=authoryear-comp, natbib=true]{biblatex} % natbib=true allows \citep
\usepackage{graphicx}
\usepackage{url}
\usepackage{hyperref}

% Section formatting using titlesec
\usepackage{titlesec}
\titleformat*{\section}{\sffamily\Large}
\titleformat*{\subsection}{\sffamily\large}
\titleformat*{\subsubsection}{\itshape}

% center figures by default
\makeatletter
\g@addto@macro\@floatboxreset\centering
\makeatother
% italic figure captions
\usepackage[format=plain,
            textfont=it]{caption}
% simple TO DO macro
\usepackage{xcolor}
\newcommand\todo[1]{\textcolor{orange}{[#1]}}

\newcommand{\indsim}{\overset{\mathrm{ind}}{\sim}}
% https://tex.stackexchange.com/questions/60545/should-i-mathrm-the-d-in-my-integrals
\newcommand*\diff{\mathop{}\!\mathrm{d}}
\newcommand*\Diff[1]{\mathop{}\!\mathrm{d^#1}}
\newcommand{\E}{\mathrm{E}}
\renewcommand{\P}{\mathrm{P}}
\newcommand{\N}{\mathrm{N}}
\newcommand{\diag}{\mathrm{diag}}
\newcommand{\ave}{\mathrm{ave}}
\renewcommand\vec{\boldsymbol}

\addbibresource{my_library.bib}
\graphicspath{{./fig/}}

\title{Title of the Paper}
\author{Leon Di Stefano}
\date{\today}

\begin{document}

% % % % % % % % % % % % % % % % % % % % % % % % % % %
\maketitle
% \vspace{0.1in}
% % % % % % % % % % % % % % % % % % % % % % % % % % %

\begin{abstract}\noindent

    Four sentences: 
    State the problem. 
    Why it's an interesting problem.
    What your solution achieves.
    What follows from your solution.
    % State the problem
    
    % Why it's an interesting problem
    
    % What your solution achieves
    
    % What follows from your solution
    
\end{abstract}

\section{Introduction}
% Describe the problem
% Use an example to introduce it
\subsubsection*{Describe the problem}

Use an example to introduce it.

% State your contributions
% Explicit list, to be backed up with evidence
\subsubsection*{State your contributions}

Use an explicit list, to be backed up by evidence.

\section{The problem}

Describe the problem, in 1 page.

\section{The main idea}


Describe your idea, in 2 pages.



\section{The idea in detail}

Peyton-Jones suggests 5 pages for this.

\section{Related work}

Put this last. 1-2 pages; be generous with credit.



\section{Conclusions and further work}

\printbibliography

\newpage

\section*{Appendix}

\end{document}
