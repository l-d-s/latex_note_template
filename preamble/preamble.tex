% fonts

\usepackage[T1]{fontenc} % Apparently this https://tex.stackexchange.com/questions/287312/font-not-loadable-metric-data-not-found-or-bad
% \usepackage[full]{textcomp}

% \usepackage{cabin} % sans serif
% \usepackage{libertine}
\usepackage{euler}
\usepackage{libertinus-type1}
\renewcommand{\familydefault}{\sfdefault}
\usepackage[libertinus,sans]{libgreek}
\usepackage[defaultmathsizes,italic]{mathastext}
\usepackage{amsmath}
\usepackage{amssymb} % For squiggly right arrow

\usepackage[varqu,varl]{inconsolata} % sans serif typewriter
% \usepackage[final,expansion=alltext]{microtype}
\usepackage[english]{babel}
% \usepackage[fleqn]{mathtools}
% \usepackage[cal=boondoxo]{mathalpha} % mathcal
\usepackage{csquotes}

% \usepackage{newtxtext}
% \usepackage{txfonts} % See https://tex.stackexchange.com/questions/444243/symbols-not-showing-up


% \usepackage[bigdelims,vvarbb]{newtxmath} % bb from STIX % "should be loaded AFTER the text font" - https://mirrors.mit.edu/CTAN/fonts/newtx/doc/newtxdoc.pdf 

% geometry of the page

\usepackage[
    top=1in,
    bottom=1in,
    left=1in,
    right=1in
    ]{geometry}

% paragraph spacing

\setlength{\parindent}{0pt}
\setlength{\parskip}{1.5ex plus 0.4ex minus 0.2ex}

% useful packages

\usepackage{mathtools} % for matrix* environment
% \usepackage{natbib} % incompatible with BibLaTeX
% \usepackage{epsfig}
% \usepackage{url}
% \usepackage{bm}
% \usepackage{blindtext}

% Leon additions

% \usepackage{enumitem} % allows for enumerate[resume]
\usepackage[style=authoryear-comp, natbib=true]{biblatex} % natbib=true allows \citep
\usepackage{graphicx}
% \usepackage{url}
% \usepackage{hyperref}

% % Section formatting using titlesec
\usepackage{titlesec}
\titleformat*{\section}{\sffamily\bfseries\Large\center}
\titleformat*{\subsection}{\sffamily\bfseries\large}
\titleformat*{\subsubsection}{\itshape}

% % center figures by default
% \makeatletter
% \g@addto@macro\@floatboxreset\centering
% \makeatother
% % italic figure captions
\usepackage[format=plain,
            textfont=it]{caption}
% % simple TO DO macro
\usepackage{xcolor}
\newcommand\todo[1]{\textcolor{orange}{[#1]}}

\newcommand{\indsim}{\overset{\mathrm{ind}}{\sim}}
% https://tex.stackexchange.com/questions/60545/should-i-mathrm-the-d-in-my-integrals
\newcommand*\diff{\mathop{}\!\mathrm{d}}
\newcommand*\Diff[1]{\mathop{}\!\mathrm{d^#1}}
\newcommand{\E}{\mathrm{E}}
\renewcommand{\P}{\mathrm{P}}
\newcommand{\N}{\mathrm{N}}
\newcommand{\diag}{\mathrm{diag}}
\newcommand{\ave}{\mathrm{ave}}
\newcommand{\corr}{\mathrm{corr}}
\newcommand{\ind}{\mathrel{\perp\mspace{-10.5mu}\perp}}
\renewcommand\vec{\boldsymbol}